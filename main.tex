\documentclass[11pt]{article}

%  USE PACKAGES  ---------------------- 
\usepackage[margin=0.7in,vmargin=1in]{geometry}
\usepackage{amsmath,amsthm,amsfonts}
\usepackage{amssymb}
\usepackage{fancyhdr}
\usepackage{enumerate}
\usepackage{mathtools}
\usepackage{hyperref,color}
\usepackage{enumitem,amssymb}
\newlist{todolist}{itemize}{4}
\setlist[todolist]{label=$\square$}
\usepackage{pifont}
\newcommand{\cmark}{\ding{51}}%
\newcommand{\xmark}{\ding{55}}%
\newcommand{\done}{\rlap{$\square$}{\raisebox{2pt}{\large\hspace{1pt}\cmark}}%
\hspace{-2.5pt}}
\newcommand{\HREF}[2]{\href{#1}{#2}}
\usepackage{textcomp}
\usepackage{listings}
\lstset{
basicstyle=\small\ttfamily,
% columns=flexible,
upquote=true,
breaklines=true,
showstringspaces=false
}
%  -------------------------------------------- 

%  HEADER AND FOOTER (DO NOT EDIT) ----------------------
\newcommand{\problemnumber}{0}
\pagestyle{fancy}
\fancyhead{}
\fancyhead[L]{\textbf{Question \problemnumber}}
\newcommand{\newquestion}[1]{
\clearpage % page break and flush floats
\renewcommand{\problemnumber}{#1} % set problem number for header
\phantom{}  % Put something on the page so it shows
}
\fancyfoot[L]{IE 332}
\fancyfoot[C]{Assignment submission}
\fancyfoot[R]{Page \thepage}
\renewcommand{\footrulewidth}{0.4pt}

%  --------------------------------------------


%  COVER SHEET (FILL IN THE TABLE AS INSTRUCTED IN THE ASSIGNMENT) ----------------------
\newcommand{\addcoversheet}{
\clearpage
\thispagestyle{empty}
\vspace*{0.5in}

\begin{center}
\Huge{{\bf IE332 Assignment 2}}

Due: April 14th, 11:59pm EST
\end{center}

\vspace{0.3in}

\noindent We have {\bf read and understood the assignment instructions}. We certify that the submitted work does not violate any academic misconduct rules, and that it is solely our own work. By listing our names below we acknowledge that any misconduct will result in appropriate consequences. 

\vspace{0.2in}

\noindent {\em ``As a Boilermaker pursuing academic excellence, I pledge to be honest and true in all that I do.
Accountable together -- we are Purdue.''}

\vspace{0.3in}

\begin{table}[h!]
  \begin{center}
    \label{tab:table1}
    \begin{tabular}{c|cc|c|c}
      Student & Q1 & Q2 & Overall & DIFF\\
      \hline
      A & 50 & 33 & 83 & 17\\
      B & 50 & 33 & 83 & 17\\
      C & 0 & 33 & 33 & -33\\
      \hline
      St Dev & 29 & 0 & 29 & 29
    \end{tabular}
  \end{center}
\end{table}

\vspace{0.2in}

\noindent Date: \today.
}
%  -----------------------------------------

%  TODO LIST (COMPLETE THE FULL CHECKLIST - USE AS EXAMPLE THE FIRST CHECKED BOXES!) ----------------------
\newcommand{\addtodo}{
\clearpage
\thispagestyle{empty}

\section*{Read Carefully. Important!}

\noindent By electronically uploading this assignment to Brightspace you acknowledge these statements and accept any repercussions if in any violation of ANY Purdue Academic Misconduct policies. You must upload your homework on time for it to be graded. No late assignments will be accepted. {\bf Only the last uploaded version of your assignment before the due date will be graded}.

\vspace{0.2in}

\noindent {\bf NOTE:} You should aim to submit no later than 30 minutes before the deadline, as there could be last minute network traffic that would cause your assignment to be late, resulting in a grade of zero. 

\vspace{0.2in}

\noindent When submitting your assignment it is assumed that every student considers the below checklist, as there are grading consequences otherwise (e.g., not submitting a cover sheet is an automatic grade of ZERO).

\begin{todolist}

    \item[\done] Your solutions were prepared using the \LaTeX template provided in Brightspace. 
    \item[\done] Your submission has a cover sheet as its first page and this checklist as its second page, according to the template provided.
	 \item All of your solutions (program code, etc.) are included in the submission as requested. % Check this checkbox and the following ones if satisfied <---
    \item You have not included any screen shots, photos, etc. (plots should be intermediately saved as .png files and then added into your .tex file). % <---
	 \item All math notation and algorithms (algorithmic environment) are created using appropriate \LaTeX code (no pictures, handwritten solutions, etc.). % <---
    \item The .pdf is submitted as an individual file and not in a {\tt .zip}.
    \item You kept the \LaTeX source code in your files until this assignment is graded, in case you are required to show proof of creating your assignment using \LaTeX.  % <---
    \item If submitting with a partner, your partner is added in the submission section in Gradescope after you upload your file. % <---
    \item You have correctly matched each question to its page \# in the .pdf submission in the Gradescope section (after you uploaded your file).
    \item Watch videos on creating pseudocode if you need a refresher or quick reference to the idea. These are good starter videos:    % <---
    
     \HREF{https://www.youtube.com/watch?v=4jLO0vXPktU}{www.youtube.com/watch?v=4jLO0vXPktU} 
    
    \HREF{https://www.youtube.com/watch?v=yGvfltxHKUU}{www.youtube.com/watch?v=yGvfltxHKUU}
\end{todolist}
}

%% LaTeX
% Für alle, die die Schönheit von Wissenschaft anderen zeigen wollen
% For anyone who wants to show the beauty of science to others

%  -----------------------------------------


\begin{document}


\addcoversheet
\addtodo

% BEGIN YOUR ASSIGNMENT HERE:

% Question 1
\newquestion{1}

\[ 
a)
In this algorithm, we have an outer loop that runs from 1 to n, and an inner loop that runs from 0 to n/i. The inner loop's time complexity depends on the value of i. As i increases, the number of iterations in the inner loop decreases. The time complexity of this algorithm can be represented as:
Σ (n/i) for i = 1 to n
This summation is equivalent to the Harmonic series, which is asymptotically bound by Θ(log n). Therefore, the time complexity of this algorithm is Θ(log n).

b)
In this algorithm, we have two nested loops. The outer loop runs from 1 to n, and the inner loop runs from 1 to √n. Thus, the outer loop works as the numerator and inner loop works as the denominator. Therefore, the time complexity of this algorithm is Θ(n√n).

c)
This algorithm is similar to (a), but the outer loop starts from 3 instead of 1. However, the difference in starting index does not affect the overall complexity. The time complexity is still dominated by the Harmonic series and is Θ(log n).

d)
In this algorithm, we have a single loop that runs until the value of X becomes less than or equal to 1. The function g(X) is called inside the loop and runs in log(n) time. The maximum number of iterations in the loop is proportional to the number of times X is multiplied by 2 or divided by 2, which is Θ(log n). Thus, the time complexity of this algorithm is Θ(log^2 n) = Θ(log n)^2, considering the loop iterations and the log(n) time of g(X).

e)
In this algorithm, we have a single loop that iterates through the elements of the list x. The loop performs different operations based on the value of i. The g(x) function is called in both cases, but it runs in different time complexities: n2 and log(n).
The time complexity of this algorithm can be represented as the sum of the time complexities of all operations. However, the maximum number of iterations in the loop cannot exceed n (the length of x). Since the loop iteration count is limited by n, the time complexity of this algorithm is dominated by the most expensive operation, which is the n^2 time complexity of g(x) in the case of even indices. Therefore, the time complexity of this algorithm is Θ(n3).

f)
The first function takes in a list of length n. It then sorts the list in ascending order by finding the minimum value in a sub-set of x and moving it to the end of the list. The subset of values starts as all of x and decreases in size by one unit each time the process is repeated.

Within the for loop, the findmin operation will have a worst case runtime of T(n) = n-i. The delete and push functions will have T(n) = 1. Therefore, one iteration of the loop will have a runtime of T(n) = n-i+2. The loop will be called n times. Therefore, the whole loop will have a runtime complexity of T(n) = n(n-i+2) = n^2 - n*i + 2*n. In the worst case, 


The second function acts as an insertion function and sorts a list of length n in ascending order. It has an outer while loop that iterates through the list x from left the right. Then, in the inner while loop, the value at the current iteration's index is moved left until the value to it's left is smaller than it's value. 


The third function acts as a bubble sort fuction and sorts a list of length n in ascending order. This function has a while loop that will iterate n times in the worst case scenario. The inner for loop will iterate n-i-1 times.




% Question 2
\newquestion{2}



\end{document}
